\section{Experiment}

\twocolsframe{Experiment Task}{
  \begin{itemize}[<+->]
    \item Tâche fondamentale : navigation et sélection.
    \item Réplication d'une tâche de classement.
    \item Grille 5x3. Classer 5 disques (en rouge) dans une cellule avec la même lettre.
    \item<+(1)-> Techniques d'interactions :
    \begin{itemize}
      \item<+(1)-> Gestes sur l'écran tactile.
      \item<+(1)-> Gestes dans l'écran virtuel.
    \end{itemize}
  \end{itemize}
}{
  \figurecaption<2>{Liu2014.jpg}{Tâche expérimentale de \cite{Liu2014}}
  \figurecaption<3>{HandheldVESADMidAirInArOut.jpg}{Notre tâche expérimentale}
  \figurecaption<4>{TaskGrid}{Grille de notre tâche expérimentale}
  \only<6>{
    \begin{figure}
      \twocols{\figuregraphic[0.7]{Wobbrock2009_1.jpg}}{\figuregraphic{Wobbrock2009_2.jpg}}
      \figuregraphic[0.6]{Wobbrock2009_3.jpg}
      \caption{Gestes de sélection, de déplacement et d'agrandissement \cite{Wobbrock2009}}
    \end{figure}
  }
  \twocolsfigure<7>{Piumsomboon2013_2.jpg}{Piumsomboon2013_3.jpg}{Gestes de sélection et de déplacement \cite{Piumsomboon2013}}
}

\twocolsframe{Experiment Design}{
  \begin{itemize}[<+->]
    \item Facteurs croisés :
    \begin{itemize}
      \item<.-> IHM :
      \begin{enumerate}
        \item Téléphone
        \item VESAD tactile
        \item VESAD
      \end{enumerate}
      \item Taille du texte : grand, petit.
      \item Distance moyenne disque-cellule : facile, difficile.
    \end{itemize}
    \item<+-> Facteur emboîté : ordre de passage IHM, 3 groupes, en carré latin.
    \item 12 participants (3 femmes, tous +18 ans, 2 de +25 ans).
  \end{itemize}
}{
  \movie<1>{ExperimentPhoneOnly.jpg}{ExperimentPhoneOnly.mp4}{(1) Téléphone}
  \movie<2>{ExperimentPhoneInArOut.jpg}{ExperimentPhoneInArOut.mp4}{(2) VESAD tactile}
  \movie<3>{ExperimentMidAirInArOut.jpg}{ExperimentMidAirInArOut.mp4}{(3) VESAD}
  \figurecaption<5-6>{TaskGrid.png}{Grille de notre tâche expérimentale}
  \only<7>{
    \begin{table}
      \scriptsize
      \begin{tabular}{| c | c | c | c |}
        \hline \textbf{Groupe} & \textbf{IHM 1} & \textbf{IHM 2} & \textbf{IHM 3}\\
        \hline 1 & Téléphone & VESAD tac. & VESAD \\
        \hline 2 & VESAD tac. & VESAD & Téléphone \\
        \hline 3 & VESAD & Téléphone & VESAD tac. \\
        \hline 
      \end{tabular}
      \caption{Ordre de passage.}
    \end{table}
  }
}