\section{Étude expérimentale}

\twocolsframe{Tâche expérimentale}{
  \begin{itemize}[<+->]
    \item Tâche fondamentale : navigation et sélection.
    \item Réplication d'une tâche de classement.
    \item Techniques d'interactions :
    \begin{itemize}
      \item<.-> Gestes sur l'écran tactile.
      \item Gestes dans l'écran virtuel.
    \end{itemize}
  \end{itemize}
}{
  \only<1>{\figurecaption{Liu2014.jpg}{Tâche de classement \cite{Liu2014}}}
  \only<2>{\figurecaption{HandheldVESADMidAirInArOut.jpg}{Notre tâche expérimentale}}
  \only<3>{
    \begin{figure}
      \twocols{\figuregraphic[0.7]{Wobbrock2009_1.jpg}}{\figuregraphic{Wobbrock2009_2.jpg}}
      \figuregraphic[0.6]{Wobbrock2009_3.jpg}
      \caption{Gestes de sélection, de déplacement et d'agrandissement \cite{Wobbrock2009}}
    \end{figure}
  }
  \only<4>{
    \begin{figure}
      \twocols{\figuregraphic{Piumsomboon2013_2.jpg}}{\figuregraphic{Piumsomboon2013_3.jpg}}
      \caption{Gestes de sélection et de déplacement \cite{Piumsomboon2013}}
    \end{figure}
  }
}

\twocolsframe{Plan expérimental}{
  \begin{itemize}
    \item<1-> Facteurs croisés :
    \begin{itemize}
      \item<1-> Techniques : (1) Téléphone, (2) VESAD tactile, (3) VESAD.
      \item<4-> Taille du texte : grand, petit.
      \item<5-> Difficulté de classement : facile, difficile.
    \end{itemize}
    \item<6-> Facteur emboîté : ordre de passage des techniques, 3 groupes, en carré latin (123, 231, 312).
    \item<7-> 12 participants (3 femmes, tous +18 ans, 2 de +25 ans).
  \end{itemize}
}{
  \only<1>{\figurecaption{ExperimentPhoneOnly.jpg}{(1) Téléphone}}
  \only<2>{\figurecaption{ExperimentPhoneInArOut.jpg}{(2) VESAD tactile}}
  \only<3>{\figurecaption{ExperimentMidAirInArOut.jpg}{(3) VESAD}}
  \only<4-5>{\figurecaption{TaskGrid.png}{La grille de la tâche expérimentale}}
  \only<6-7>{\figurecaption{HandheldVESADMidAirInArOut.jpg}{Notre tâche expérimentale}}
}

\twocolsframe{Résultats}{
  \begin{itemize}[<+->]
    \item Barres d'erreurs : intervalles de confiances à 95\%.
    \item Temps de complétion
    \begin{itemize}
      \item Effet de la technique : VESAD tactile < Téléphone < VESAD.
      \item Effet du groupe : apprentissage.
    \end{itemize}
    \item Erreurs et sélections
    \begin{itemize}
      \item .
    \end{itemize}
    \item Navigation
    \begin{itemize}
      \item .
    \end{itemize}
    \item Notes des participants
    \begin{itemize}
      \item .
    \end{itemize}
  \end{itemize}
}{
  \only<3-4>{\figurecaption{tct.png}{Temps de complétion moyen (moyenne arithmétique).}}
  \only<5>{\figurecaption{selections.png}{}}
  \only<6>{\figurecaption{errors.png}{}}
  \only<7>{\figurecaption{preferences_distribution.png}{}}
}