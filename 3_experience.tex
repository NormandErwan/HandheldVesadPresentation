\section{Étude expérimentale}

\twocolsframe{Tâche expérimentale}{
  \begin{itemize}[<+->]
    \item Tâche fondamentale : navigation et sélection.
    \item Réplication d'une tâche de classement.
    \item Grille 5x3. Classer 5 disques (en rouge) dans une cellule avec la même lettre.
    \item Techniques d'interactions :
    \begin{itemize}
      \item<.-> Gestes sur l'écran tactile.
      \item Gestes dans l'écran virtuel.
    \end{itemize}
  \end{itemize}
}{
  \figurecaption<2>{Liu2014.jpg}{Tâche expérimentale de \cite{Liu2014}}
  \figurecaption<3>{HandheldVESADMidAirInArOut.jpg}{Notre tâche expérimentale}
  \only<4>{
    \begin{figure}
      \twocols{\figuregraphic[0.7]{Wobbrock2009_1.jpg}}{\figuregraphic{Wobbrock2009_2.jpg}}
      \figuregraphic[0.6]{Wobbrock2009_3.jpg}
      \caption{Gestes de sélection, de déplacement et d'agrandissement \cite{Wobbrock2009}}
    \end{figure}
  }
  \twocolsfigure<5>{Piumsomboon2013_2.jpg}{Piumsomboon2013_3.jpg}{Gestes de sélection et de déplacement \cite{Piumsomboon2013}}
}

\twocolsframe{Plan expérimental}{
  \begin{itemize}[<+->]
    \item Facteurs croisés :
    \begin{itemize}
      \item<.-> IHM :
      \begin{enumerate}
        \item Téléphone
        \item VESAD tactile
        \item VESAD
      \end{enumerate}
      \item Taille du texte : grand, petit.
      \item Difficulté (distance moyenne disque-cellule) : facile, difficile.
    \end{itemize}
    \item<+-> Facteur emboîté : ordre de passage IHM, 3 groupes, en carré latin.
  \end{itemize}
}{
  \externalmovie<2>{ExperimentPhoneOnly.jpg}{(1) Téléphone}{ExperimentPhoneOnly.mp4}
  \externalmovie<3>{ExperimentPhoneInArOut.jpg}{(2) VESAD tactile}{ExperimentPhoneInArOut.mp4}
  \externalmovie<4>{ExperimentMidAirInArOut.jpg}{(3) VESAD}{ExperimentMidAirInArOut.mp4}
  \figurecaption<5-6>{TaskGrid}{Grille de la tâche}
  \only<7>{
    \begin{table}
      \scriptsize
      \begin{tabular}{| c | c | c | c |}
        \hline \textbf{Groupe} & \textbf{IHM 1} & \textbf{IHM 2} & \textbf{IHM 3}\\
        \hline 1 & Téléphone & VESAD tac. & VESAD \\
        \hline 2 & VESAD tac. & VESAD & Téléphone \\
        \hline 3 & VESAD & Téléphone & VESAD tac. \\
        \hline 
      \end{tabular}
      \caption{Ordre de passage.}
    \end{table}
  }
}

\twocolsframe{Résultats}{
  \begin{itemize}[<+->]
    \item 12 participants (3 femmes, tous +18 ans, 2 de +25 ans).
    \item Barres d'erreurs : intervalles de confiances à 95\%.
    \item Temps de complétion
    \begin{itemize}
      \item Effet de l'IHM :
      \begin{enumerate}
        \item<.-> VESAD tactile
        \item<.-> Téléphone (+22s)
        \item<.-> VESAD (+49s)
      \end{enumerate}
      \item Effet du groupe : car apprentissage.
    \end{itemize}
  \end{itemize}
}{
  \figurecaption<4>{tct}{Temps de complétion moyen}
  \figurecaption<5-7>{tct_ordering}{Temps de complétion moyen par groupe}
  \framezoom<5><6>(6.5cm,0.5cm)(3.7cm,3.7cm)
}

\twocolsframe{Résultats}{
  \begin{itemize}[<+->]
    \item Erreurs (disque mal classé) : pas de différences.
    \item Sélections :
    \begin{itemize}
      \item Correlé aux erreurs.
      \item Effet de l'IHM : un peu plus sur Téléphone.
      \item Effet du groupe : car apprentissage.
    \end{itemize}
  \end{itemize}
}{
  \figurecaption<1>{errors}{Erreurs moyennes}
  \figurecaption<4>{selections}{Sélections moyennes de disques}
  \figurecaption<5>{selections_ordering}{Sélections moyennes de disques par groupe}
}

\twocolsframe{Résultats}{
  \begin{itemize}[<+->]
    \item Navigation
    \begin{itemize}
      \item VESAD : navigation physique.
      \item Téléphone et VESAD tactile :
      \begin{itemize}
        \item 5-6 défilements par disque.
        \item<.-> Téléphone : 3 zooms par disque.
        \item<.-> VESAD tactile : 1 zoom par disque.
        \item Différences similaires en temps.
      \end{itemize}
    \end{itemize}
  \end{itemize}
}{
  \figurecaption<1-4>{navigation_count}{} \framezoom<1><2>(6.5cm,0.5cm)(4cm,3.7cm)
  \figurecaption<5-6>{navigation_time}{}  \framezoom<5><6>(6.5cm,0cm)(4cm,3.5cm)
}

\twocolsframe{Résultats}{
  \begin{itemize}[<+->]
    \item Notes des participants :
    \begin{itemize}
      \item Facile à comprendre.
      \item Mentalement facile à utiliser.
      \item Physiquement facile à utiliser.
      \item Rapidité perçue.
      \item Performance perçue.
      \item Frustration.
      \item Préférences.
    \end{itemize}
  \end{itemize}
}{
  \figurecaption<3>{easy_understand.png}{Facile à comprendre.}
  \figurecaption<4>{mentally_easy.png}{Mentalement facile à utiliser.}
  \figurecaption<5>{physically_easy.png}{Physiquement facile à utiliser.}
  \figurecaption<6>{speed.png}{Rapidité perçue.}
  \figurecaption<7>{performance.png}{Performance perçue.}
  \figurecaption<8>{frustration.png}{Frustration.}
  \figurecaption<9>{preferences.png}{Préférences.}
}