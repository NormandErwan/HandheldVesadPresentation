% !TEX root
% !TEX program = latexmk

% Largement inspiré par l'excellent exemple français de présentation de Till Tantau (2004) et Philippe de Sousa (2006) : https://github.com/josephwright/beamer/blob/master/doc/solutions/conference-talks/conference-ornate-20min.fr.tex

\documentclass[aspectratio=169]{beamer}

\usepackage{lmodern}
\usepackage[french]{babel}
\usepackage[utf8]{inputenc}
\usepackage[T1]{fontenc}

\setbeamertemplate{caption}{\raggedright\insertcaption\par} % Suppression du Figure ou Table préfixé dans les \caption par beamer :  https://tex.stackexchange.com/a/82460

\mode<presentation> {
  \usetheme{default}
  \useinnertheme{circles}
  \useoutertheme[subsections=false]{smoothbars}
  \usecolortheme[rgb={0.8,0,0}]{structure}
}

\AtBeginSection[] {
  \begin{frame}<beamer>{Plan de la soutenance}
    \tableofcontents[currentsection]
  \end{frame}
}

\title{Agrandissement d'un écran mobile par réalité augmentée}
\author{Erwan Normand}
\institute{Soutenance de mémoire de maîtrise}
\date{\frenchdate{2018}{08}{29}}

\newcommand{\twocols}[2]{% Two side-by-side colums
  \begin{columns}%
    \begin{column}{0.5\textwidth}#1\end{column}%
    \begin{column}{0.5\textwidth}#2\end{column}%
  \end{columns}%
}
\newcommand{\twocolsframe}[3]{%
  \begin{frame}{#1}%
    \twocols{#2}{#3}%
  \end{frame}%
}
\newcommand{\figuregraphic}[2][1]{%
  \includegraphics[width=#1\textwidth, height=5cm, keepaspectratio]{figures/#2}%
}
\newcommand{\figurecaption}[3][1]{%
  \vspace{-1cm}
  \begin{figure}%
    \figuregraphic[#1]{#2}
    \caption{#3}%
  \end{figure}%
}

\begin{document}

\frame{\maketitle}

\begin{frame}{Plan de la soutenance}
  \tableofcontents
\end{frame}

\section{Problématique}

\begin{frame}{La réalité augmentée (RA)}
  \twocols{
    \begin{itemize}[<+->]
      \item Définition : contenu virtuel inséré dans l'environnement réel.
      \item Avancées techniques en 2016/2017 : HoloLens, ARKit, ARCore.
      \item Usages très prometteurs.
      \item Les interfaces humain-machine en RA sont peu maîtrisées.
    \end{itemize}
  }{
    \only<1>{\figurecaption{MobileAR}{Application de RA sur téléphone [Wikipedia]}}
    \only<2>{\figurecaption{HoloLens_1}{Publicité du visiocasque HoloLens [Microsoft]}}
    \only<3>{\figurecaption{HoloLens_2}{Skype sur le HoloLens [Microsoft]}}
    \only<4>{\figurecaption{UnityFutureMRPartIII2017}{Bureau de travail en RA [Unity]}}
  }
\end{frame}

\begin{frame}{Les interfaces humain-machine (IHM)}
  \twocols{
    \begin{itemize}[<+->]
      \item IHM : interface d'un utilisateur avec un ordinateur.
      \item Trouver les techniques d'interactions les plus adaptées.
      \item Réduire l'écart entre entrées et sorties.
    \end{itemize}
  }{
    \figurecaption{Billinghurst2005}{Principe d'une IHM \cite{Billinghurst2005}}
  }
\end{frame}

\begin{frame}{IHM pour la RA}
  \twocols{
    \begin{itemize}
      \item<1-> La RA permettrait de fondre l'ordinateur dans l'environnement.
      \item<2-> Mais encore aucun paradigme d'IHM pour la RA.
      \item<3-> Difficile car entrées \alert{et} sorties sont en 3D.
      \item<4-> Techniques d'interactions courantes :
      \begin{itemize}
        \item Main virtuelle
        \item<5-> Pointeur virtuel
      \end{itemize}
    \end{itemize}
  }{
    \only<1-2>{\figurecaption{Serrano2015_1}{Gluey \cite{Serrano2015}}}
    \only<3>{\figurecaption{Lee2013}{SpaceTop \cite{Lee2013}}}
    \only<4>{\figurecaption{Taylor2016}{Main virtuelle \cite{Taylor2016}}}
    \only<5>{\figurecaption{Pfeuffer2017_1}{Pointeur virtuel \cite{Pfeuffer2017}}}
  }
\end{frame}

\begin{frame}{Augmenter un téléphone par RA}
  \twocols{
    \begin{itemize}[<+->]
      \item Les téléphones ont une taille limitée pour être tenu en main.
      \item Un plus grand écran serait utile.
      \item Les interactions tactiles sont précises et connues.
      \item Des interactions sur un écran virtuel pourraient être intuives mais difficiles.
    \end{itemize}
  }{
    \only<1,3>{\figurecaption{Heun2016}{Reality Editor \cite{Heun2016}}}
    \only<2>{\figurecaption{Baudisch2002}{Écran agrandi \cite{Baudisch2002}}}
    \only<4>{\figurecaption{LeapMotion2018_3}{Visiocasque de RA North Star [LeapMotion]}}
  }
\end{frame}

\begin{frame}{Problématique}
  \begin{block}<1->{}
     Est-ce qu’un téléphone à écran étendu donne un avantage à un utilisateur par rapport à un téléphone non-étendu ?
  \end{block}
  \begin{block}<2->{}
    Est-il préférable d’interagir avec une main virtuelle directement sur l’écran virtuel, autour du téléphone, ou en utilisant seulement l’écran tactile ?
  \end{block}
\end{frame}
\section{Prototype de visiocasque de RA}

\twocolsframe{Conception}{
  \begin{itemize}[<+->]
    \item Besoin : visiocasque à large champs de vision.
    \item Solution :
    \begin{itemize}
      \item<.-> Diffuser une caméra stéréoscopique dans un visiocasque de réalité virtuelle (RV).
      \item Réplication de l'AR-Rift \cite{Steptoe2013}.
    \end{itemize}
    \item Contribution : 
    \begin{itemize}
      \item Documenter la conception.
      \item Produire la bibliothèque ArucoUnity.
    \end{itemize}
  \end{itemize}
}{
  \only<1>{\figurecaption{Fov}{Champs de vision}}
  \only<2>{\figurecaption{ArRift_1}{Notre AR-Rift}}
  \only<3->{\figurecaption{ArRift_2}{Notre AR-Rift porté}}
}

\twocolsframe{Fonctionnement}{
  \begin{enumerate}[<+->]
    \item Capture de l'image avec la caméra physique.
    \item Correction de l'image.
    \item Une caméra virtuelle filme les éléments virtuels et l'image corrigée en arrière-plan.
    \item Diffusion du résultat dans le visiocasque de RV.
  \end{enumerate}
}{
  \only<1>{\figurecaption{ArRiftMarker_1}{Image fisheye}}
  \only<2>{\figurecaption{ArRiftMarker_2}{Image fisheye corrigée}}
  \only<3>{\figurecaption{ArRiftMarker_3}{Caméra virtuelle filmant la scène et l'image corrigée}}
  \only<4>{\figurecaption{ArRiftMarker_4}{Résultat dans le visiocasque}}
}

\twocolsframe{ArucoUnity}{
  \begin{itemize}[<+->]
    \item Greffon pour le moteur 3D Unity.
    \item<.-> Utilise la bibliothèque libre OpenCV.
    \item Prise en charge des caméras stéréoscopique, fisheye. 
    \item Étalonnage de la caméra.
    \item Suivi de marqueurs en temps réel.
  \end{itemize}
}{
  \only<1>{
    \begin{figure}
      \twocols{\figuregraphic{UnityLogo}}{\figuregraphic[0.8]{OpenCvLogo}}
      \caption{Logos d'Unity et d'OpenCV}
    \end{figure}
  }
  \only<2>{\figurecaption{OvrvisionTracking_2}{RA dans notre visiocasque}}
  \only<3>{\figurecaption{OvrvisionCalibration_3}{Étalonnage de notre camera}}
  \only<4->{
    \begin{figure}
      \twocols{\figuregraphic{TrackMarkersDemo_2}}{\figuregraphic{TrackMarkersDemo_3}}
      \caption{RA par suivi de marqueur}
    \end{figure}
  }
}
\section{Étude expérimentale}

\twocolsframe{Tâche expérimentale}{
  \begin{itemize}[<+->]
    \item Tâche fondamentale : navigation et sélection.
    \item Réplication d'une tâche de classement.
    \item Grille 5x3. Classer 5 disques (en rouge) dans une cellule avec la même lettre.
    \item Techniques d'interactions :
    \begin{itemize}
      \item<.-> Gestes sur l'écran tactile.
      \item Gestes dans l'écran virtuel.
    \end{itemize}
  \end{itemize}
}{
  \figurecaption<2>{Liu2014.jpg}{Tâche expérimentale de \cite{Liu2014}}
  \figurecaption<3>{HandheldVESADMidAirInArOut.jpg}{Notre tâche expérimentale}
  \only<4>{
    \begin{figure}
      \twocols{\figuregraphic[0.7]{Wobbrock2009_1.jpg}}{\figuregraphic{Wobbrock2009_2.jpg}}
      \figuregraphic[0.6]{Wobbrock2009_3.jpg}
      \caption{Gestes de sélection, de déplacement et d'agrandissement \cite{Wobbrock2009}}
    \end{figure}
  }
  \twocolsfigure<5>{Piumsomboon2013_2.jpg}{Piumsomboon2013_3.jpg}{Gestes de sélection et de déplacement \cite{Piumsomboon2013}}
}

\twocolsframe{Plan expérimental}{
  \begin{itemize}[<+->]
    \item Facteurs croisés :
    \begin{itemize}
      \item<.-> IHM :
      \begin{enumerate}
        \item Téléphone
        \item VESAD tactile
        \item VESAD
      \end{enumerate}
      \item Taille du texte : grand, petit.
      \item Difficulté (distance moyenne disque-cellule) : facile, difficile.
    \end{itemize}
    \item<+-> Facteur emboîté : ordre de passage IHM, 3 groupes, en carré latin.
  \end{itemize}
}{
  \externalmovie<2>{ExperimentPhoneOnly.jpg}{(1) Téléphone}{ExperimentPhoneOnly.mp4}
  \externalmovie<3>{ExperimentPhoneInArOut.jpg}{(2) VESAD tactile}{ExperimentPhoneInArOut.mp4}
  \externalmovie<4>{ExperimentMidAirInArOut.jpg}{(3) VESAD}{ExperimentMidAirInArOut.mp4}
  \figurecaption<5-6>{TaskGrid}{Grille de la tâche}
  \only<7>{
    \begin{table}
      \scriptsize
      \begin{tabular}{| c | c | c | c |}
        \hline \textbf{Groupe} & \textbf{IHM 1} & \textbf{IHM 2} & \textbf{IHM 3}\\
        \hline 1 & Téléphone & VESAD tac. & VESAD \\
        \hline 2 & VESAD tac. & VESAD & Téléphone \\
        \hline 3 & VESAD & Téléphone & VESAD tac. \\
        \hline 
      \end{tabular}
      \caption{Ordre de passage.}
    \end{table}
  }
}

\twocolsframe{Résultats}{
  \begin{itemize}[<+->]
    \item 12 participants (3 femmes, tous +18 ans, 2 de +25 ans).
    \item Barres d'erreurs : intervalles de confiances à 95\%.
    \item Temps de complétion
    \begin{itemize}
      \item Effet de l'IHM :
      \begin{enumerate}
        \item<.-> VESAD tactile
        \item<.-> Téléphone (+22s)
        \item<.-> VESAD (+49s)
      \end{enumerate}
      \item Effet du groupe : car apprentissage.
    \end{itemize}
  \end{itemize}
}{
  \figurecaption<4>{tct}{Temps de complétion moyen}
  \figurecaption<5-7>{tct_ordering}{Temps de complétion moyen par groupe}
  \framezoom<5><6>(6.5cm,0.5cm)(3.7cm,3.7cm)
}

\twocolsframe{Résultats}{
  \begin{itemize}[<+->]
    \item Erreurs (disque mal classé) : pas de différences.
    \item Sélections :
    \begin{itemize}
      \item Correlé aux erreurs.
      \item Effet de l'IHM : un peu plus sur Téléphone.
      \item Effet du groupe : car apprentissage.
    \end{itemize}
  \end{itemize}
}{
  \figurecaption<1>{errors}{Erreurs moyennes}
  \figurecaption<4>{selections}{Sélections moyennes de disques}
  \figurecaption<5>{selections_ordering}{Sélections moyennes de disques par groupe}
}

\twocolsframe{Résultats}{
  \begin{itemize}[<+->]
    \item Navigation
    \begin{itemize}
      \item VESAD : navigation physique.
      \item Téléphone et VESAD tactile :
      \begin{itemize}
        \item 5-6 défilements par disque.
        \item<.-> Téléphone : 3 zooms par disque.
        \item<.-> VESAD tactile : 1 zoom par disque.
        \item Différences similaires en temps.
      \end{itemize}
    \end{itemize}
  \end{itemize}
}{
  \figurecaption<1-4>{navigation_count}{} \framezoom<1><2>(6.5cm,0.5cm)(4cm,3.7cm)
  \figurecaption<5-6>{navigation_time}{}  \framezoom<5><6>(6.5cm,0cm)(4cm,3.5cm)
}

\twocolsframe{Résultats}{
  \begin{itemize}[<+->]
    \item Notes des participants :
    \begin{itemize}
      \item Facile à comprendre.
      \item Mentalement facile à utiliser.
      \item Physiquement facile à utiliser.
      \item Rapidité perçue.
      \item Performance perçue.
      \item Frustration.
      \item Préférences.
    \end{itemize}
  \end{itemize}
}{
  \figurecaption<3>{easy_understand.png}{Facile à comprendre.}
  \figurecaption<4>{mentally_easy.png}{Mentalement facile à utiliser.}
  \figurecaption<5>{physically_easy.png}{Physiquement facile à utiliser.}
  \figurecaption<6>{speed.png}{Rapidité perçue.}
  \figurecaption<7>{performance.png}{Performance perçue.}
  \figurecaption<8>{frustration.png}{Frustration.}
  \figurecaption<9>{preferences.png}{Préférences.}
}
\section{Discussion du concept}

\twocolsframe{Retour sur l'étude expérimentale}{
  \begin{itemize}[<+->]
    \item Problématique :
    \begin{itemize}
      \item Les participants ont préféré l'écran étendu au téléphone seul.
      \item Les interactions tactiles ont été plus performantes que la main virtuelle.
      \item Certaines tâches bénéfieraient donc d'un écran étendu.
    \end{itemize}
    \item Limites :
    \begin{itemize}
      \item Difficulté mal conçue.
      \item Suivi de la main trop peu fiable.
      \item Mauvaise résolution du visiocasque.
    \end{itemize}
  \end{itemize}
}{
  \figurecaption{HandheldVESADMidAirInArOut.jpg}{Notre tâche expérimentale}
}

\twocolsframe{Concevoir une IHM pour un VESAD}{
  \begin{itemize}[<+->]
    \item Recommandations :
    \begin{itemize}
      \item Penser l'IHM à travers l'interaction.
      \item L'IHM s'adapte au contexte.
      \item Main virtuelle : le suivi est continu, besoin d'actions discrètes.
    \end{itemize}
    \item Exemple d'applications :
    \begin{itemize}
      \item En vue multi-fenêtres.
      \item En vue étendue.
    \end{itemize}
    \item Techniques d'interactions :
    \begin{itemize}
      \item Wrist : basculer entre les applications.
      \item Slide-to-hang : détacher une fenêtre de l'écran.
    \end{itemize}
  \end{itemize}
}{
  \twocolsfigure<2-3>{HandheldVESADzones_1.png}{HandheldVESADzones_2.png}{Gauche : main virtuelle. Droite : écran tactile. Zones d'interactions (vert, jaune) et visualisation seulement (blanc).}
  \figurecaption<4>{LeapMotion2018_3.jpg}{Visiocasque de RA North Star [LeapMotion]}
  \figurecaption<6>{HandheldVESADApps3.jpg}{Applications en vue multi-fenêtres}
  \twocolsfigure<7>{HandheldVESADMap.jpg}{HandheldVESADDogs.jpg}{Applications en vue étendue}
  \twocolsfigure<9>{HandheldVESADApps.jpg}{HandheldVESADApps2.jpg}{Wrist}
  \figurecaption<10>{HandheldVESADSlideToHang.jpg}{Slide-to-hang}
}

\twocolsframe{Directions futures}{
  \begin{itemize}
    \item Évaluer un pointeur virtuel.
    \item Tester des applications réelles.
    \item Déterminer taille idéale écran étendu.
    \item Explorer changements de contextes.
  \end{itemize}
}{
  \figurecaption<1>{Pfeuffer2017_1.jpg}{Pointeur virtuel \cite{Pfeuffer2017}}
  \twocolsfigure<2->{HandheldVESADApps3.jpg}{HandheldVESADMap.jpg}{Applications sur un VESAD mobile}
}
\section{Conclusion}

\twocolsframe{Conclusion}{
  \begin{itemize}[<+->]
    \item VESAD : écran étendu par RA.
    \item Réalisations :
    \begin{itemize}
      \item Visiocasque de RA à large champ de vision.
      \item Bibliothèque de RA ArucoUnity.
    \end{itemize}
    \item Étude expérimentale :
    \begin{itemize}
      \item Écran étendu contre téléphone seul.
      \item Interactions tactiles contre main virtuelle.
      \item VESAD tactile le plus performant et préféré des participants.
    \end{itemize}
    \item Potentiel à confirmer sur applications concrètes.
  \end{itemize}
}{
  \figurecaption{HandheldVESADMidAirInArOut.jpg}{Notre tâche expérimentale}
}

\appendix
\section<presentation>*{\appendixname}
\begin{frame}[allowframebreaks]{Bibliographie}
  \begin{thebibliography}{10}
    \bibitem[Billinghurst2005]{Billinghurst2005}
      Mark Billinghurst, Raphael Grasset, Julian Looser.
      \newblock Designing augmented reality interfaces.
      \newblock ACM SIGGRAPH Computer Graphics, 39(1), 17-22.

    \bibitem[Ens2014]{Ens2014}
      Barrett M. Ens, Rory Finnegan, Pourang P. Irani.
      \newblock The personal cockpit: A spatial interface for effective task switching on head-worn displays.
      \newblock CHI 2014.

    \bibitem[Grubert2015]{Grubert2015}
      Jens Grubert, Matthias Heinisch, Aaron J. Quigley, Dieter Schmalstieg.
      \newblock MultiFi: Multi fidelity interaction with displays on and around the body.
      \newblock CHI 2015.

    \bibitem[Heun2016]{Heun2016}
      Valentin Heun, James Hobin, Pattie Maes.
      \newblock Reality Editor: Programming Smarter Objects.
      \newblock UbiComp 2013.

    \bibitem[Lee2013]{Lee2013}
      Jinh Lee, Alex Olwal, Hiroshi Ishii, Cati Boulanger.
      \newblock SpaceTop: Integrating 2D and spatial 3D interactions in a see-through desktop environment.
      \newblock CHI 2013.

    \bibitem[Liu2014]{Liu2014}
      Can Liu, Olivier Chapuis, Michel Beaudouin-Lafon, Eric Lecolinet, Wendy E. Mackay.
      \newblock Effects of display size and navigation type on a classification task.
      \newblock CHI 2014.

    \bibitem[Piumsomboon2013]{Piumsomboon2013}
      Thammathip Piumsomboon, Adrian Clark, Mark Billinghurst, Andy Cockburn.
      \newblock User-defined gestures for augmented reality.
      \newblock CHI 2013.

    \bibitem[Pfeuffer2017]{Pfeuffer2017}
      Ken Pfeuffer, Benedikt Mayer, Diako Mardanbegi, Hans Gellersen.
      \newblock Gaze + pinch interaction in virtual reality.
      \newblock UIST 2017.

    \bibitem[Serrano2015]{Serrano2015}
      Marcos Serrano, Barrett Ens, Xing-Dong Yang, Pourang Irani.
      \newblock Gluey: Developing a head-worn display interface to unify the interaction experience in distributed display environments.
      \newblock MobileHCI 2015.

    \bibitem[Serrano2015a]{Serrano2015a}
      Marcos Serrano, Barrett Ens, Xing-Dong Yang, Pourang Irani.
      \newblock Desktop-Gluey: Augmenting desktop environments with wearable devices.
      \newblock MobileHCI 2015.

    \bibitem[Steptoe2013]{Steptoe2013}
      William Steptoe.
      \newblock AR-Rift (Part 1).
      \newblock \url{http://willsteptoe.com/post/66968953089/ar-rift-part-1}.

    \bibitem[Taylor2016]{Taylor2016}
      Jonathan Taylor, Lucas Bordeaux, Thomas J. Cashman, Bob Corish, Cem Keskin, Toby Sharp, Eduardo Soto, David Sweeney, Julien P. C. Valentin, Benjamin Luff, Arran Topalian, Erroll Wood, Sameh Khamis, Pushmeet Kohli, Shahram Izadi, Richard Banks, Andrew W. Fitzgibbon, Jamie Shotton.
      \newblock Efficient and precise interactive hand tracking through joint, continuous optimization of pose and correspondences.
      \newblock ACM Transactions on Graphics, 35(4), 1-12.

    \bibitem[Wobbrock2009]{Wobbrock2009}
      Jacob O. Wobbrock and Meredith Ringel Morris and Andrew D. Wilson.
      \newblock User-defined gestures for surface computing.
      \newblock CHI 2009.
  \end{thebibliography}
\end{frame}

\end{document}