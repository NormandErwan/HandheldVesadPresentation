% !TEX root
% !TEX program = latexmk

% Largement inspiré par l'excellent exemple français de présentation de Till Tantau (2004) et Philippe de Sousa (2006) : https://github.com/josephwright/beamer/blob/master/doc/solutions/conference-talks/conference-ornate-20min.fr.tex

\documentclass[aspectratio=169]{beamer}

\usepackage{lmodern}
\usepackage[french]{babel}
\usepackage[utf8]{inputenc}
\usepackage[T1]{fontenc}

\setbeamertemplate{caption}{\raggedright\insertcaption\par} % Suppression du Figure ou Table préfixé dans les \caption par beamer :  https://tex.stackexchange.com/a/82460

\mode<presentation> {
  \usetheme{default}
  \useinnertheme{circles}
  \useoutertheme[subsections=false]{smoothbars}
  \usecolortheme[rgb={0.8,0,0}]{structure}
}

\AtBeginSection[] {
  \begin{frame}<beamer>{Plan de la soutenance}
    \tableofcontents[currentsection]
  \end{frame}
}

\title{Agrandissement d'un écran mobile par réalité augmentée}
\author{Erwan Normand}
\institute{Soutenance de mémoire de maîtrise}
\date{\frenchdate{2018}{08}{29}}

\newcommand{\twocols}[2]{% Two side-by-side colums
  \begin{columns}%
    \begin{column}{0.5\textwidth}#1\end{column}%
    \begin{column}{0.5\textwidth}#2\end{column}%
  \end{columns}%
}
\newcommand{\twocolsframe}[3]{%
  \begin{frame}{#1}%
    \twocols{#2}{#3}%
  \end{frame}%
}
\newcommand{\figuregraphic}[2][1]{%
  \includegraphics[width=#1\textwidth, height=5cm, keepaspectratio]{figures/#2}%
}
\newcommand{\figurecaption}[3][1]{%
  \vspace{-1cm}
  \begin{figure}%
    \figuregraphic[#1]{#2}
    \caption{#3}%
  \end{figure}%
}

\begin{document}

\frame{\maketitle}

\begin{frame}{Plan de la soutenance}
  \tableofcontents
\end{frame}

\section{Problématique}

\twocolsframe{La réalité augmentée (RA)}{
  \begin{itemize}[<+->]
    \item Définition : contenu virtuel inséré en temps réel dans l'environnement réel.
    \item Avancées techniques en 2016/2017 : HoloLens, ARKit, ARCore.
    \item Usages très prometteurs.
    \item Les interfaces humain-machine en RA sont peu maîtrisées.
  \end{itemize}
}{
  \figurecaption<1>{MobileAR.jpg}{Application de RA sur téléphone [Wikipedia]}
  \figurecaption<2>{HoloLens_1.jpg}{Publicité du visiocasque HoloLens [Microsoft]}
  \figurecaption<3>{HoloLens_2.jpg}{Visioconférence sur le HoloLens [Microsoft]}
  \figurecaption<4>{UnityFutureMRPartIII2017.jpg}{Bureau de travail en RA [Unity]}
}

\twocolsframe{Les interfaces humain-machine (IHM)}{
  \begin{itemize}[<+->]
    \item IHM : interface d'un utilisateur avec un ordinateur.
    \item Trouver les techniques d'interactions les plus adaptées.
    \item Réduire l'écart entre entrées et sorties.
  \end{itemize}
}{
  \figurecaption<1->{Billinghurst2005.png}{Principe d'une IHM \cite{Billinghurst2005}}
}

\twocolsframe{Les IHM pour la RA}{
  \begin{itemize}
    \item<1-> La RA permettrait de fondre l'ordinateur avec l'environnement.
    \item<2-> Les IHM de RA sont encore dans un stade exploratoire.
    \item<3-> Difficile car IHM en 3D.
    \item<4-> Techniques d'interactions courantes :
    \begin{itemize}
      \item Main virtuelle
      \item<5-> Pointeur virtuel
    \end{itemize}
  \end{itemize}
}{
  \figurecaption<1>{Serrano2015_1.jpg}{Gluey \cite{Serrano2015}}
  \figurecaption<2>{Serrano2015a_1.jpg}{Desktop-Gluey \cite{Serrano2015a}}
  \figurecaption<3>{Lee2013.jpg}{SpaceTop \cite{Lee2013}}
  \figurecaption<4>{Taylor2016.jpg}{Main virtuelle \cite{Taylor2016}}
  \figurecaption<5>{Pfeuffer2017_1.jpg}{Pointeur virtuel \cite{Pfeuffer2017}}
}

\twocolsframe{Augmenter un téléphone par RA}{
  \begin{itemize}[<+->]
    \item Les téléphones ont une taille limitée pour être tenu en main.
    \item La RA permet de s'entourer de multiples écrans.
    \item Un plus grand écran serait utile.
    \item Concept : Virtually Extended Screen-Aligned Display (VESAD).
    \item Quelles interactions ?
    \begin{itemize}
      \item Les interactions tactiles sont précises, fiables, stables, connues et tangibles.
      \item Des interactions sur un écran virtuel pourraient être intuitives mais difficiles.
    \end{itemize}
  \end{itemize}
}{
  \figurecaption<1>{Heun2016.jpg}{Reality Editor \cite{Heun2016}}
  \figurecaption<2>{Ens2014_3.jpg}{Personal Cockpit \cite{Ens2014}}
  \figurecaption<3>{Grubert2015_2.jpg}{MultiFi \cite{Grubert2015}}
  \figurecaption<4>{HandheldVESADApps.jpg}{Grille d'applications sur un VESAD mobile}
  \figurecaption<5>{Serrano2015a_3.jpg}{Desktop-Gluey \cite{Serrano2015a}}
  \figurecaption<6>{LeapMotion2018_3.jpg}{Visiocasque de RA North Star [LeapMotion]}
}

\begin{frame}{Problématique}
  \begin{enumerate}[<+(1)->]
    \item Est-ce qu’un téléphone à écran étendu donne un avantage à un utilisateur par rapport à un téléphone non-étendu ?
    \item Est-il préférable d’interagir avec une main virtuelle directement sur l’écran virtuel, autour du téléphone, ou en utilisant seulement l’écran tactile ?
  \end{enumerate}
\end{frame}
\section{Prototype de visiocasque de RA}

\twocolsframe{Conception}{
  \begin{itemize}[<+->]
    \item Besoin : visiocasque à large champs de vision.
    \item Solution :
    \begin{itemize}
      \item<.-> Diffuser une caméra stéréoscopique dans un visiocasque de réalité virtuelle (RV).
      \item Réplication de l'AR-Rift \cite{Steptoe2013}.
    \end{itemize}
    \item Contribution : 
    \begin{itemize}
      \item Documenter la conception.
      \item Produire la bibliothèque ArucoUnity.
    \end{itemize}
  \end{itemize}
}{
  \figurecaption<1>{Fov.png}{Champs de vision}
  \figurecaption<2>{ArRift_1.jpg}{Notre AR-Rift}
  \figurecaption<3->{ArRift_2.jpg}{Notre AR-Rift porté}
}

\twocolsframe{Fonctionnement}{
  \begin{enumerate}[<+->]
    \item Capture de l'image avec la caméra physique.
    \item Correction de l'image.
    \item Une caméra virtuelle filme les éléments virtuels et l'image corrigée en arrière-plan.
    \item Diffusion du résultat dans le visiocasque de RV.
  \end{enumerate}
}{
  \figurecaption<1>{ArRiftMarker_1.jpg}{Image fisheye}
  \figurecaption<2>{ArRiftMarker_2.jpg}{Image fisheye corrigée}
  \figurecaption<3>{ArRiftMarker_3.jpg}{Caméra virtuelle filmant la scène et l'image corrigée}
  \figurecaption<4>{ArRiftMarker_4.jpg}{Résultat dans le visiocasque}
}

\twocolsframe{ArucoUnity}{
  \begin{itemize}[<+->]
    \item Greffon pour le moteur 3D Unity.
    \item<.-> Utilise la bibliothèque libre OpenCV.
    \item Prise en charge des caméras stéréoscopique, fisheye. 
    \item Étalonnage de la caméra.
    \item Suivi de marqueurs en temps réel.
  \end{itemize}
}{
  \twocolsfigure<1>{UnityLogo.png}{OpenCvLogo.png}{Logos d'Unity et d'OpenCV}
  \figurecaption<2>{OvrvisionTracking_2.jpg}{RA dans notre visiocasque}
  \figurecaption<3>{OvrvisionCalibration_3.jpg}{Étalonnage de notre camera}
  \twocolsfigure<4->{TrackMarkersDemo_2.jpg}{TrackMarkersDemo_3.jpg}{RA par suivi de marqueur}
}
\section{Étude expérimentale}

\twocolsframe{Tâche expérimentale}{
  \begin{itemize}[<+->]
    \item Tâche fondamentale : navigation et sélection.
    \item Réplication d'une tâche de classement.
    \item Techniques d'interactions :
    \begin{itemize}
      \item<.-> Gestes sur l'écran tactile.
      \item Gestes dans l'écran virtuel.
    \end{itemize}
  \end{itemize}
}{
  \only<1>{\figurecaption{Liu2014}{Tâche de classement \cite{Liu2014}}}
  \only<2>{\figurecaption{HandheldVESADMidAirInArOut}{Notre tâche expérimentale}}
  \only<3>{
    \begin{figure}
      \twocols{\figuregraphic[0.7]{Wobbrock2009_1}}{\figuregraphic{Wobbrock2009_2}}
      \figuregraphic[0.6]{Wobbrock2009_3}
      \caption{Gestes de sélection, de déplacement et d'agrandissement \cite{Wobbrock2009}}
    \end{figure}
  }
  \only<4>{
    \begin{figure}
      \twocols{\figuregraphic{Piumsomboon2013_2}}{\figuregraphic{Piumsomboon2013_3}}
      \caption{Gestes de sélection et de déplacement \cite{Piumsomboon2013}}
    \end{figure}
  }
}

\twocolsframe{Plan expérimental}{
  \begin{itemize}
    \item<1-> Facteurs croisés :
    \begin{itemize}
      \item<1-> IHM : (1) téléphone seul, (2) VESAD tactile, (3) VESAD.
      \item<4-> Taille du texte : grand, petit.
      \item<5-> Difficulté de classement : facile, difficile.
    \end{itemize}
    \item<6-> Facteur emboîté : ordre de passage des IHM, 3 groupes, en carré latin (123, 231, 312).
    \item<7-> 12 participants (3 femmes, tous +18 ans, 2 de +25 ans).
  \end{itemize}
}{
  \only<1>{\figurecaption{ExperimentPhoneOnly}{(1) Téléphone seul}}
  \only<2>{\figurecaption{ExperimentPhoneInArOut}{(2) VESAD tactile}}
  \only<3>{\figurecaption{ExperimentMidAirInArOut}{(3) VESAD}}
  \only<4-5>{\figurecaption{TaskGrid}{Notre tâche expérimentale}}
  \only<6-7>{\figurecaption{HandheldVESADMidAirInArOut}{Notre tâche expérimentale}}
}

\twocolsframe{Résultats}{
  \begin{itemize}[<+->]
    \item Les barres d'erreurs sont des intervalles de confiances à 95\%.
    \item Temps de complétion
    \begin{itemize}
      \item 
    \end{itemize}
    \item Erreurs et sélections
    \begin{itemize}
      \item .
    \end{itemize}
    \item Navigation
    \begin{itemize}
      \item .
    \end{itemize}
    \item Notes des participants
    \begin{itemize}
      \item .
    \end{itemize}
  \end{itemize}
}{
  \only<2>{\figurecaption{tct}{Temps de complétion moyen (moyenne arithmétique).}
  \only<3>{\figurecaption{selections}{}
  \only<4>{\figurecaption{errors}{}
  \only<5>{\figurecaption{preferences_distribution}{}
}
\section{Discussion du concept}

\twocolsframe{Concept}{
  \begin{itemize}[<+->]
    \item Première contribution : explorer l'IHM d'un écran
    \item Écran étendu ou VESAD (Virtually Extended Screen-Aligned Display).
  \end{itemize}
}{
  \only<1>{\figurecaption{HandheldVESADApps}{Applications en vue multi-fenêtres}}
  \only<2>{
    \begin{figure}
      \twocols{\figuregraphic{HandheldVESADMap}}{\figuregraphic{HandheldVESADDogs}}
      \caption{Applications en vue étendue}
    \end{figure}
  }
}

\twocolsframe{Techniques d'interactions}{
  \begin{itemize}
    \item<1-> Wrist : pour basculer entre les applications.
    \item<2-> Slide-to-hang : détacher une fenêtre de l'écran.
  \end{itemize}
}{
  \only<1>{
    \begin{figure}
      \twocols{\figuregraphic{HandheldVESADApps}}{\figuregraphic{HandheldVESADApps2}}
      \caption{Wrist}
    \end{figure}
  }
  \only<2>{\figurecaption{HandheldVESADSlideToHang}{Slide-to-hang}}
}
\input{5_conclusion}

\appendix
\section<presentation>*{\appendixname}
\begin{frame}[allowframebreaks]{Bibliographie}
  \begin{thebibliography}{10}
    \bibitem[Billinghurst \textit{et al.} 2005]{Billinghurst2005}
      Mark Billinghurst, Raphael Grasset, Julian Looser.
      \newblock Designing augmented reality interfaces.
      \newblock ACM SIGGRAPH Computer Graphics, 39(1), 17-22.

    \bibitem[Ens \textit{et al.} 2014]{Ens2014}
      Barrett M. Ens, Rory Finnegan, Pourang P. Irani.
      \newblock The personal cockpit: A spatial interface for effective task switching on head-worn displays.
      \newblock CHI 2014.

    \bibitem[Grubert \textit{et al.} 2015]{Grubert2015}
      Jens Grubert, Matthias Heinisch, Aaron J. Quigley, Dieter Schmalstieg.
      \newblock MultiFi: Multi fidelity interaction with displays on and around the body.
      \newblock CHI 2015.

    \bibitem[Heun \textit{et al.} 2016]{Heun2016}
      Valentin Heun, James Hobin, Pattie Maes.
      \newblock Reality Editor: Programming Smarter Objects.
      \newblock UbiComp 2013.

    \bibitem[Lee \textit{et al.} 2013]{Lee2013}
      Jinh Lee, Alex Olwal, Hiroshi Ishii, Cati Boulanger.
      \newblock SpaceTop: Integrating 2D and spatial 3D interactions in a see-through desktop environment.
      \newblock CHI 2013.

    \bibitem[Liu \textit{et al.} 2014]{Liu2014}
      Can Liu, Olivier Chapuis, Michel Beaudouin-Lafon, Eric Lecolinet, Wendy E. Mackay.
      \newblock Effects of display size and navigation type on a classification task.
      \newblock CHI 2014.

    \bibitem[Piumsomboon \textit{et al.} 2013]{Piumsomboon2013}
      Thammathip Piumsomboon, Adrian Clark, Mark Billinghurst, Andy Cockburn.
      \newblock User-defined gestures for augmented reality.
      \newblock CHI 2013.

    \bibitem[Pfeuffer \textit{et al.} 2017]{Pfeuffer2017}
      Ken Pfeuffer, Benedikt Mayer, Diako Mardanbegi, Hans Gellersen.
      \newblock Gaze + pinch interaction in virtual reality.
      \newblock UIST 2017.

    \bibitem[Serrano \textit{et al.} 2015]{Serrano2015}
      Marcos Serrano, Barrett Ens, Xing-Dong Yang, Pourang Irani.
      \newblock Gluey: Developing a head-worn display interface to unify the interaction experience in distributed display environments.
      \newblock MobileHCI 2015.

    \bibitem[Serrano \textit{et al.} 2015a]{Serrano2015a}
      Marcos Serrano, Barrett Ens, Xing-Dong Yang, Pourang Irani.
      \newblock Desktop-Gluey: Augmenting desktop environments with wearable devices.
      \newblock MobileHCI 2015.

    \bibitem[Steptoe2013]{Steptoe2013}
      William Steptoe.
      \newblock AR-Rift (Part 1).
      \newblock \url{http://willsteptoe.com/post/66968953089/ar-rift-part-1}.

    \bibitem[Taylor \textit{et al.} 2016]{Taylor2016}
      Jonathan Taylor, Lucas Bordeaux, Thomas J. Cashman, Bob Corish, Cem Keskin, Toby Sharp, Eduardo Soto, David Sweeney, Julien P. C. Valentin, Benjamin Luff, Arran Topalian, Erroll Wood, Sameh Khamis, Pushmeet Kohli, Shahram Izadi, Richard Banks, Andrew W. Fitzgibbon, Jamie Shotton.
      \newblock Efficient and precise interactive hand tracking through joint, continuous optimization of pose and correspondences.
      \newblock ACM Transactions on Graphics, 35(4), 1-12.

    \bibitem[Wobbrock \textit{et al.} 2009]{Wobbrock2009}
      Jacob O. Wobbrock and Meredith Ringel Morris and Andrew D. Wilson.
      \newblock User-defined gestures for surface computing.
      \newblock CHI 2009.
  \end{thebibliography}
\end{frame}

\end{document}