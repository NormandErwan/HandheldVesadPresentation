% Largement inspiré par l'excellent exemple français de présentation de Till Tantau (2004) et Philippe de Sousa (2006) : https://github.com/josephwright/beamer/blob/master/doc/solutions/conference-talks/conference-ornate-20min.fr.tex

\usepackage{lmodern}
\usepackage[french]{babel}
\usepackage[utf8]{inputenc}
\usepackage[T1]{fontenc}

\setbeamertemplate{caption}{\raggedright\insertcaption\par} % Suppression du Figure ou Table préfixé dans les \caption par beamer :  https://tex.stackexchange.com/a/82460

\mode<presentation> {
  \usetheme{default}
  \useinnertheme{circles}
  \useoutertheme[subsections=false]{smoothbars}
  \usecolortheme[rgb={0.8,0,0}]{structure}
}

\AtBeginSection[] {
  \begin{frame}<beamer>{Plan de la soutenance}
    \tableofcontents[currentsection]
  \end{frame}
}

\title{Agrandissement d'un écran mobile par réalité augmentée}
\author{Erwan Normand}
\institute{Soutenance de mémoire de maîtrise}
\date{\frenchdate{2018}{08}{29}}

\newcommand{\twocols}[2]{% Side-By-Side colums
  \begin{columns}%
    \begin{column}{0.5\textwidth}#1\end{column}%
    \begin{column}{0.5\textwidth}#2\end{column}%
  \end{columns}%
}
\newcommand{\captionfigure}[3][1]{%
  \begin{figure}%
    \includegraphics[width=#1\textwidth, height=5cm, keepaspectratio]{figures/#2}%
    \caption{#3}%
  \end{figure}%
}

\begin{document}

\frame{\maketitle}

\begin{frame}{Plan de la soutenance}
  \tableofcontents
\end{frame}

\section{Problématique}

\begin{frame}{La réalité augmentée (RA)}
  \twocols{
    \begin{itemize}[<+->]
      \item Définition : contenu virtuel inséré dans l'environnement réel.
      \item Avancées techniques en 2016/2017 : HoloLens, ARKit, ARCore.
      \item Usages très prometteurs.
      \item Les interfaces humain-machine en RA sont peu maîtrisées.
    \end{itemize}
  }{
    \only<1>{\figurecaption{MobileAR}{Application de RA sur téléphone [Wikipedia]}}
    \only<2>{\figurecaption{HoloLens_1}{Publicité du visiocasque HoloLens [Microsoft]}}
    \only<3>{\figurecaption{HoloLens_2}{Skype sur le HoloLens [Microsoft]}}
    \only<4>{\figurecaption{UnityFutureMRPartIII2017}{Bureau de travail en RA [Unity]}}
  }
\end{frame}

\begin{frame}{Les interfaces humain-machine (IHM)}
  \twocols{
    \begin{itemize}[<+->]
      \item IHM : interface d'un utilisateur avec un ordinateur.
      \item Trouver les techniques d'interactions les plus adaptées.
      \item Réduire l'écart entre entrées et sorties.
    \end{itemize}
  }{
    \figurecaption{Billinghurst2005}{Principe d'une IHM \cite{Billinghurst2005}}
  }
\end{frame}

\begin{frame}{IHM pour la RA}
  \twocols{
    \begin{itemize}
      \item<1-> La RA permettrait de fondre l'ordinateur dans l'environnement.
      \item<2-> Mais encore aucun paradigme d'IHM pour la RA.
      \item<3-> Difficile car entrées \alert{et} sorties sont en 3D.
      \item<4-> Techniques d'interactions courantes :
      \begin{itemize}
        \item Main virtuelle
        \item<5-> Pointeur virtuel
      \end{itemize}
    \end{itemize}
  }{
    \only<1-2>{\figurecaption{Serrano2015_1}{Gluey \cite{Serrano2015}}}
    \only<3>{\figurecaption{Lee2013}{SpaceTop \cite{Lee2013}}}
    \only<4>{\figurecaption{Taylor2016}{Main virtuelle \cite{Taylor2016}}}
    \only<5>{\figurecaption{Pfeuffer2017_1}{Pointeur virtuel \cite{Pfeuffer2017}}}
  }
\end{frame}

\begin{frame}{Augmenter un téléphone par RA}
  \twocols{
    \begin{itemize}[<+->]
      \item Les téléphones ont une taille limitée pour être tenu en main.
      \item Un plus grand écran serait utile.
      \item Les interactions tactiles sont précises et connues.
      \item Des interactions sur un écran virtuel pourraient être intuives mais difficiles.
    \end{itemize}
  }{
    \only<1,3>{\figurecaption{Heun2016}{Reality Editor \cite{Heun2016}}}
    \only<2>{\figurecaption{Baudisch2002}{Écran agrandi \cite{Baudisch2002}}}
    \only<4>{\figurecaption{LeapMotion2018_3}{Visiocasque de RA North Star [LeapMotion]}}
  }
\end{frame}

\begin{frame}{Problématique}
  \begin{block}<1->{}
     Est-ce qu’un téléphone à écran étendu donne un avantage à un utilisateur par rapport à un téléphone non-étendu ?
  \end{block}
  \begin{block}<2->{}
    Est-il préférable d’interagir avec une main virtuelle directement sur l’écran virtuel, autour du téléphone, ou en utilisant seulement l’écran tactile ?
  \end{block}
\end{frame}
\section{Travaux reliés}

\begin{frame}{Le Personal Cockpit}
  \twocols{
  }{
    \figurecaption{Ens2014_3}{Le Personal Cockpit \cite{Ens2014}}
  }
\end{frame}
\section{Design Concept}

\begin{frame}{Our work: VESAD (Virtually Extended Screen Aligned Display)}
  \twocolsfigure{HandheldVESADApps.jpg}{App list}{HandheldVESADMap.jpg}{Map}{}
  \vspace{-30px}\framefootnote{Mockups.}
\end{frame}

\begin{frame}{Our work: VESAD (Virtually Extended Screen Aligned Display)}
  \twocolsfigure{HandheldVESADTooltip.jpg}{Tooltips}{HandheldVESADDogs.jpg}{Photo gallery}{}
  \vspace{-30px}\framefootnote{Mockups.}
\end{frame}

\begin{frame}{Our work: VESAD (Virtually Extended Screen Aligned Display)}
  \twocolsfigure{HandheldVESADMap.jpg}{}{HandheldVESADApps2.jpg}{}{Wrist}
  \vspace{-25px}\framefootnote{Mockup.}
\end{frame}

\begin{frame}{Our work: VESAD (Virtually Extended Screen Aligned Display)}
  \figurecaption{HandheldVESADSlideToHang.jpg}{Slide-to-hang}
  \vspace{-25px}\framefootnote{Mockup.}
\end{frame}
\section{Prototype}

\begin{frame}{Titre}
  contenu
\end{frame}
\section{Étude expérimentale}

\begin{frame}{Titre}
  contenu
\end{frame}
\section{Discussion}

\begin{frame}{Titre}
  contenu
\end{frame}
\section{Conclusion}

\begin{frame}{Conclusion}
\end{frame}

\appendix
\section<presentation>*{\appendixname}
\begin{frame}[allowframebreaks]{Bibliographie}
  \begin{thebibliography}{10}
    \bibitem[Ens2014]{Ens2014}
      Barrett M. Ens and Rory Finnegan and Pourang P. Irani.
      \newblock The personal cockpit: A spatial interface for effective task switching on head-worn displays.
      \newblock CHI 2014 (pp. 3171--3180).

      \bibitem[Heun2016]{Heun2016}
        Heun, Valentin and Hobin, James and Maes, Pattie.
        \newblock Reality Editor: Programming Smarter Objects.
        \newblock UbiComp 2013 (pp. 307--310).
  \end{thebibliography}
\end{frame}

\end{document}