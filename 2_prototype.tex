\section{Prototype de visiocasque de RA}

\twocolsframe{Conception}{
  \begin{itemize}[<+->]
    \item Besoins :
    \begin{itemize}
      \item Visiocasque à large champ de vision.
      \item Suivre le téléphone en 3D.
    \end{itemize}
    \item Solutions :
    \begin{itemize}
      \item Diffuser une caméra stéréoscopique fisheye dans un visiocasque de réalité virtuelle (RV) \cite{Steptoe2013}.
      \item Utiliser des marqueurs.
    \end{itemize}
    \item Contributions :
    \begin{itemize}
      \item Documenter la conception du visiocasque.
      \item Produire la bibliothèque ArucoUnity.
    \end{itemize}
  \end{itemize}
}{
  \figurecaption<5>{ArRift_1.jpg}{Notre AR-Rift}
  \figurecaption<6>{ExperimentSmartphone.jpg}{Marqueurs autour du téléphone}
}

\twocolsframe{Fonctionnement}{
  \begin{enumerate}[<+->]
    \item Capture de l'image avec la caméra physique.
    \item Correction de l'image : les lignes sont droites.
    \item Une caméra virtuelle filme les éléments virtuels et l'image corrigée en arrière-plan.
    \item Diffusion du résultat dans le visiocasque de RV.
  \end{enumerate}
}{
  \figurecaption<1>{ArRiftMarker_1.jpg}{Image fisheye}
  \figurecaption<2>{ArRiftMarker_2.jpg}{Image fisheye corrigée}
  \figurecaption<3>{ArRiftMarker_3.jpg}{Caméra virtuelle filmant la scène et l'image corrigée}
  \figurecaption<4>{ArRiftMarker_4.jpg}{Résultat dans le visiocasque}
}

\twocolsframe{ArucoUnity}{
  \begin{itemize}[<+->]
    \item Greffon pour le moteur 3D Unity.
    \item<.-> Utilise la bibliothèque libre OpenCV.
    \item Prise en charge des caméras stéréoscopiques et des objectifs fisheyes.
    \item Suivi de marqueurs en temps réel.
    \item Travail directement dans Unity sans avoir à coder.
  \end{itemize}
}{
  \twocolsfigure<1>{UnityLogo.png}{OpenCvLogo.png}{Logos d'Unity et d'OpenCV}
  \figurecaption<2>{OvrvisionTracking_2.jpg}{RA dans notre visiocasque}
  \twocolsfigure<3->{TrackMarkersDemo_2.jpg}{TrackMarkersDemo_3.jpg}{RA par suivi de marqueur}
}