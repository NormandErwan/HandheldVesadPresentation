\section{Prototype de visiocasque de RA}

\twocolsframe{Conception}{
  \begin{itemize}[<+->]
    \item Besoin : visiocasque à large champs de vision.
    \item Solution :
    \begin{itemize}
      \item<.-> Diffuser une caméra stéréoscopique dans un visiocasque de réalité virtuelle (RV).
      \item Réplication de l'AR-Rift \cite{Steptoe2013}.
    \end{itemize}
    \item Contribution : 
    \begin{itemize}
      \item Documenter la conception.
      \item Produire la bibliothèque ArucoUnity.
    \end{itemize}
  \end{itemize}
}{
  \only<1>{\figurecaption{Fov}{Champs de vision}}
  \only<2>{\figurecaption{ArRift_1}{Notre AR-Rift}}
  \only<3->{\figurecaption{ArRift_2}{Notre AR-Rift porté}}
}

\twocolsframe{Fonctionnement}{
  \begin{enumerate}[<+->]
    \item Capture de l'image avec la caméra physique.
    \item Correction de l'image.
    \item Une caméra virtuelle filme les éléments virtuels et l'image corrigée en arrière-plan.
    \item Diffusion du résultat dans le visiocasque de RV.
  \end{enumerate}
}{
  \only<1>{\figurecaption{ArRiftMarker_1}{Image fisheye}}
  \only<2>{\figurecaption{ArRiftMarker_2}{Image fisheye corrigée}}
  \only<3>{\figurecaption{ArRiftMarker_3}{Caméra virtuelle filmant la scène et l'image corrigée}}
  \only<4>{\figurecaption{ArRiftMarker_4}{Résultat dans le visiocasque}}
}

\twocolsframe{ArucoUnity}{
  \begin{itemize}[<+->]
    \item Greffon pour le moteur 3D Unity.
    \item<.-> Utilise la bibliothèque libre OpenCV.
    \item Prise en charge des caméras stéréoscopique, fisheye. 
    \item Étalonnage de la caméra.
    \item Suivi de marqueurs en temps réel.
  \end{itemize}
}{
  \only<1>{
    \begin{figure}
      \twocols{\figuregraphic{UnityLogo}}{\figuregraphic[0.8]{OpenCvLogo}}
      \caption{Logos d'Unity et d'OpenCV}
    \end{figure}
  }
  \only<2>{\figurecaption{OvrvisionTracking_2}{RA dans notre visiocasque}}
  \only<3>{\figurecaption{OvrvisionCalibration_3}{Étalonnage de notre camera}}
  \only<4->{
    \begin{figure}
      \twocols{\figuregraphic{TrackMarkersDemo_2}}{\figuregraphic{TrackMarkersDemo_3}}
      \caption{RA par suivi de marqueur}
    \end{figure}
  }
}