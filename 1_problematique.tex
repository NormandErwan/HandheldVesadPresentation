\section{Problématique}

\begin{frame}{La réalité augmentée (RA)}
  \twocols{
    \begin{itemize}
      \item<1-> Définition : contenu virtuel inséré dans l'environnement réel.
      \item<2-> Avancées techniques en 2016/2017 : HoloLens, ARKit, ARCore.
      \item<3-> Usages très prometteurs.
      \item<4-> Les IHMs en RA sont peu maîtrisées.
    \end{itemize}
  }{
    \only<1>{\captionfigure{MobileAR}{Application de RA sur téléphone (Wikipedia).}}
    \only<2>{\captionfigure{HoloLens_1}{Publicité du visiocasque HoloLens (Microsoft).}}
    \only<3>{\captionfigure{HoloLens_2}{Skype sur le HoloLens (Microsoft).}}
    \only<4>{\captionfigure{UnityFutureMRPartIII2017}{Bureau de travail en RA (Unity).}}
  }
\end{frame}

\begin{frame}{Combinaison de la RA avec téléphone}
  \twocols{
    \begin{itemize}
      \item<1-> Les téléphones ont une taille limitée pour être tenu en main.
      \item<2-> Un plus grand écran serait utile.
      \item<3-> Les interactions tactiles sont précises et connues.
      \item<4-> Des interactions sur un écran virtuel pourraient être intuives mais difficiles.
    \end{itemize}
  }{
    \only<1,3>{\captionfigure{Heun2016}{Reality Editor. \cite{Heun2016}}}
    \only<2>{\captionfigure{Baudisch2002}{Écran agrandi. \cite{Baudisch2002}}}
    \only<4>{\captionfigure{LeapMotion2018_3}{Visiocasque de RA North Star (Leap Motion).}}
  }
\end{frame}

\begin{frame}{Problématique}
  \begin{block}{}
     Est-ce qu’un téléphone à écran étendu donne un avantage à un utilisateur par rapport à un téléphone non-étendu ?
  \end{block}
  \begin{block}{}
    Est-il préférable d’interagir avec une main virtuelle directement sur l’écran virtuel, autour du téléphone, ou en utilisant seulement l’écran tactile ?
  \end{block}
\end{frame}