\section{Problématique}

\twocolsframe{La réalité augmentée (RA)}{
  \begin{itemize}[<+->]
    \item Définition : contenu virtuel inséré dans l'environnement réel.
    \item Avancées techniques en 2016/2017 : HoloLens, ARKit, ARCore.
    \item Usages très prometteurs.
    \item Les interfaces humain-machine en RA sont peu maîtrisées.
  \end{itemize}
}{
  \figurecaption<1>{MobileAR.jpg}{Application de RA sur téléphone [Wikipedia]}
  \figurecaption<2>{HoloLens_1.jpg}{Publicité du visiocasque HoloLens [Microsoft]}
  \figurecaption<3>{HoloLens_2.jpg}{Skype sur le HoloLens [Microsoft]}
  \figurecaption<4>{UnityFutureMRPartIII2017.jpg}{Bureau de travail en RA [Unity]}
}

\twocolsframe{Les interfaces humain-machine (IHM)}{
  \begin{itemize}[<+->]
    \item IHM : interface d'un utilisateur avec un ordinateur.
    \item Trouver les techniques d'interactions les plus adaptées.
    \item Réduire l'écart entre entrées et sorties.
  \end{itemize}
}{
  \figurecaption<1->{Billinghurst2005.png}{Principe d'une IHM \cite{Billinghurst2005}}
}

\twocolsframe{Les IHM pour la RA}{
  \begin{itemize}
    \item<1-> La RA permettrait de fondre l'ordinateur avec l'environnement.
    \item<2-> Mais encore aucun paradigme d'IHM pour la RA.
    \item<3-> Difficile car entrées \alert{et} sorties sont en 3D.
    \item<4-> Techniques d'interactions courantes :
    \begin{itemize}
      \item Main virtuelle
      \item<5-> Pointeur virtuel
    \end{itemize}
  \end{itemize}
}{
  \figurecaption<1>{Serrano2015_1.jpg}{Gluey \cite{Serrano2015}}
  \figurecaption<2>{Serrano2015a_1.jpg}{Desktop-Gluey \cite{Serrano2015a}}
  \figurecaption<3>{Lee2013.jpg}{SpaceTop \cite{Lee2013}}
  \figurecaption<4>{Taylor2016.jpg}{Main virtuelle \cite{Taylor2016}}
  \figurecaption<5>{Pfeuffer2017_1.jpg}{Pointeur virtuel \cite{Pfeuffer2017}}
}

\twocolsframe{Augmenter un téléphone par RA}{
  \begin{itemize}[<+->]
    \item Les téléphones ont une taille limitée pour être tenu en main.
    \item Un plus grand écran serait utile.
    \item La RA permet de s'entourer de multiples écrans.
    \item Les interactions tactiles sont précises et connues.
    \item Des interactions sur un écran virtuel pourraient être intuives mais difficiles.
  \end{itemize}
}{
  \figurecaption<1>{Heun2016.jpg}{Reality Editor \cite{Heun2016}}
  \figurecaption<2>{Grubert2015_2.jpg}{MultiFi \cite{Grubert2015}}
  \figurecaption<3>{Ens2014_2.jpg}{Personal Cockpit \cite{Ens2014}}
  \figurecaption<4>{Serrano2015a_3.jpg}{Desktop-Gluey \cite{Serrano2015a}}
  \figurecaption<5>{LeapMotion2018_3.jpg}{Visiocasque de RA North Star [LeapMotion]}
}

\begin{frame}{Problématique}
  \begin{block}{}
    Nom du concept : Virtually Extended Screen-Aligned Display (VESAD) / écran étendu.    
  \end{block}
  \begin{block}{}
    \begin{enumerate}[<+(1)->]
      \item Est-ce qu’un téléphone à écran étendu donne un avantage à un utilisateur par rapport à un téléphone non-étendu ?
      \item Est-il préférable d’interagir avec une main virtuelle directement sur l’écran virtuel, autour du téléphone, ou en utilisant seulement l’écran tactile ?
    \end{enumerate}
  \end{block}
\end{frame}