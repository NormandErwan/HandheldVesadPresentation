\section{Discussion du concept}

\twocolsframe{Retour sur l'étude expérimentale}{
  \begin{itemize}[<+->]
    \item Problématique :
    \begin{enumerate}
      \item Les participants ont préféré l'écran étendu au téléphone seul.
      \item Les interactions tactiles ont été plus performantes que la main virtuelle.
    \end{enumerate}
    \item Certaines tâches bénéfieraient donc d'un écran étendu.
    \item Limites :
    \begin{itemize}
      \item Facteurs taille et distance trop semblables.
      \item Suivi de la main trop peu fiable.
      \item Mauvaise résolution du visiocasque.
    \end{itemize}
  \end{itemize}
}{
  \figurecaption{HandheldVESADMidAirInArOut.jpg}{Notre tâche expérimentale}
}

\twocolsframe{Concevoir une IHM pour un VESAD}{
  \begin{itemize}[<+->]
    \item Recommandations :
    \begin{itemize}
      \item Interactions dans certaines zones.
      \item Adapter l'interface au contexte d'utilisation.
    \end{itemize}
    \item Exemple d'applications :
    \begin{itemize}
      \item En vue multi-fenêtres.
      \item En vue étendue.
    \end{itemize}
    \item Techniques d'interactions :
    \begin{itemize}
      \item Wrist : basculer entre les applications.
      \item Slide-to-hang : détacher une fenêtre de l'écran.
    \end{itemize}
  \end{itemize}
}{
  \twocolsfigure<2-3>{HandheldVESADzones_1.png}{HandheldVESADzones_2.png}{Gauche : main virtuelle. Droite : écran tactile. Zones d'interactions (vert, jaune) et visualisation seulement (blanc).}
  \figurecaption<4>{HandheldVESADApps3.jpg}{Applications en vue multi-fenêtres}
  \twocolsfigure<5>{HandheldVESADMap.jpg}{HandheldVESADDogs.jpg}{Applications en vue étendue}
  \twocolsfigure<7>{HandheldVESADApps.jpg}{HandheldVESADApps2.jpg}{Wrist}
  \figurecaption<8>{HandheldVESADSlideToHang.jpg}{Slide-to-hang}
}

\twocolsframe{Directions futures}{
  \begin{itemize}[<+->]
    \item Évaluer un pointeur virtuel.
    \item Améliorer la main virtuelle.
    \item Évaluer des applications réelles.
    \item Déterminer taille idéale écran étendu.
    \item Explorer changements de contextes.
  \end{itemize}
}{
  \figurecaption<1>{Pfeuffer2017_1.jpg}{Pointeur virtuel \cite{Pfeuffer2017}}
  \figurecaption<2>{LeapMotion2018_3.jpg}{Visiocasque de RA North Star [LeapMotion]}
  \twocolsfigure<3->{HandheldVESADApps3.jpg}{HandheldVESADMap.jpg}{Applications sur un VESAD mobile}
}