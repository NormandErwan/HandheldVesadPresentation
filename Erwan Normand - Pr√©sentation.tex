% Largement inspiré par l'excellent exemple français de présentation de Till Tantau (2004) et Philippe de Sousa (2006) : https://github.com/josephwright/beamer/blob/master/doc/solutions/conference-talks/conference-ornate-20min.fr.tex

\documentclass{beamer}

\usepackage{lmodern}
\usepackage[french]{babel}
\usepackage[utf8]{inputenc}
\usepackage[T1]{fontenc}

\mode<presentation> {
  \usetheme{default}
  \useinnertheme{circles}
  \useoutertheme[subsections=false]{smoothbars}
  \usecolortheme[rgb={0.9,0,0}]{structure}
}

\title[Agrandissement d'un écran mobile par RA]{Agrandissement d'un écran mobile par réalité augmentée}
\author[Erwan Normand]{Erwan Normand}% \\ \small{\texttt{normand.erwan@protonmail.com}}}
%\institute{École de Technologie Supérieure}
\institute{Soutenance de mémoire de maîtrise}
\date{\frenchdate{2018}{08}{29}}

\begin{document}

\maketitle

\begin{frame}{Plan de la soutenance}
  \tableofcontents
\end{frame}


\section{Problématique} % Chapitre d'introduction
\begin{frame}{Titre}
  contenu
\end{frame}


\section{Travaux reliés} % Chapitre de revue de littérature
\begin{frame}{Titre}
  contenu
\end{frame}


\section{Concept} % Chapitre du concept
\begin{frame}{Titre}
  contenu
\end{frame}


\section{Prototype} % Chapitre du visiocasque
\begin{frame}{Titre}
  contenu
\end{frame}


\section{Étude expérimentale} % Chapitre étude expérimentale
\begin{frame}{Titre}
  contenu
\end{frame}


\section{Discussion} % Chapitre discussion
\begin{frame}{Titre}
  contenu
\end{frame}

\section*{Conclusion}
\begin{frame}{Conclusion}
  \begin{itemize}
    \item Le \alert{premier message principal} de l'exposé en une ligne ou deux.
    \item Le \alert{deuxième message principal} de l'exposé en une ligne ou deux.
    \item Peut-être un \alert{troisième message}, mais pas plus que ça.
  \end{itemize}
\end{frame}


\appendix
\section<presentation>*{\appendixname}
\begin{frame}{Bibliographie}
\end{frame}

\end{document}