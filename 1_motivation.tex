\section{Research Problem}

\begin{frame}{Motivation}
  \movie{Iphones.jpg}{Iphones.mp4}{UnlockRiver.com. ALL iPhones Compared!. https://youtu.be/vG7m30AwPKQ}
\end{frame}

\begin{frame}{Motivation}
  \movie{LeapMotion2018.jpg}{LeapMotion2018.mp4}{Leap Motion. Project North Star: Desk UI. https://youtu.be/6dB1IRg3Qls}
\end{frame}

\twocolsframe{Enlarge a phone screen with AR}{
  \begin{itemize}[<+->]
    \item Les téléphones ont une taille limitée pour être tenu en main.
    \item La RA permet de s'entourer de multiples écrans.
    \item Un plus grand écran serait utile.
    \item Concept : Virtually Extended Screen-Aligned Display (VESAD).
    \item Quelles interactions ?
    \begin{itemize}
      \item Les interactions tactiles sont précises, fiables, stables, connues et tangibles.
      \item Des interactions sur un écran virtuel pourraient être intuitives mais difficiles.
    \end{itemize}
    \item Research questions:
    \begin{enumerate}[<+(1)->]
      \item Est-ce qu’un téléphone à écran étendu donne un avantage à un utilisateur par rapport à un téléphone non-étendu ?
      \item Est-il préférable d’interagir avec une main virtuelle directement sur l’écran virtuel, autour du téléphone, ou en utilisant seulement l’écran tactile ?
    \end{enumerate}
  \end{itemize}
}{
  \figurecaption<1>{Heun2016.jpg}{Reality Editor \cite{Heun2016}}
  \figurecaption<2>{Ens2014_3.jpg}{Personal Cockpit \cite{Ens2014}}
  \figurecaption<3>{Grubert2015_2.jpg}{MultiFi \cite{Grubert2015}}
  \figurecaption<4>{HandheldVESADApps.jpg}{Grille d'applications sur un VESAD mobile}
  \figurecaption<6>{Serrano2015a_3.jpg}{Desktop-Gluey \cite{Serrano2015a}}
  \figurecaption<7>{LeapMotion2018_3.jpg}{Visiocasque de RA North Star [LeapMotion]}
}