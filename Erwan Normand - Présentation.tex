\documentclass{beamer}

% Largement inspiré par l'excellent exemple français de présentation de Till Tantau (2004) et Philippe de Sousa (2006) : https://github.com/josephwright/beamer/blob/master/doc/solutions/conference-talks/conference-ornate-20min.fr.tex

\mode<presentation> {
  \usetheme{Warsaw}
}

\usepackage{lmodern}
\usepackage[french]{babel}
\usepackage[utf8]{inputenc}
\usepackage[T1]{fontenc}

\title[Agrandissement d'un écran mobile par RA]{Agrandissement d'un écran mobile par réalité augmentée}
\author{Erwan Normand \\ \texttt{normand.erwan@protonmail.com}}
\institute{École de Technologie Supérieure}
\date{\frenchdate{2018}{08}{29}}

\AtBeginSubsection[] {
  \begin{frame}<beamer>{Plan de la soutenance}
    \tableofcontents[currentsection,currentsubsection]
  \end{frame}
}

\begin{document}

\maketitle

\begin{frame}{Plan de la soutenance}
  \tableofcontents
\end{frame}


\section{Problème de recherche}
\subsection{Problématique} % Chapitre d'introduction

\subsection{Travaux reliés} % Chapitre de revue de littérature


\section{Contributions}
\subsection{Concept} % Chapitre du concept

\subsection{Prototype} % Chapitre du visiocasque

\subsection{Étude expérimentale} % Chapitre étude expérimentale


\section{Discussion} % Chapitre discussion


\section*{Conclusion}
\begin{frame}{Conclusion}
  \begin{itemize}
    \item Le \alert{premier message principal} de l'exposé en une ligne ou deux.
    \item Le \alert{deuxième message principal} de l'exposé en une ligne ou deux.
    \item Peut-être un \alert{troisième message}, mais pas plus que ça.
  \end{itemize}
\end{frame}


\appendix
\section<presentation>*{\appendixname}
\subsection<presentation>*{Bibliographie}

\begin{frame}[allowframebreaks]{Bibliographie}
  \begin{thebibliography}{10}
  \end{thebibliography}
\end{frame}

\end{document}